\chapter{Evolutionary modeling of the differential contribution of Neanderthal ancestry to complex traits provides insights into selective forces that shape trait variation}
\section{Introduction}
We recently developed a methodology to assess whether Neanderthal ancestry is over- or under-represented in the genetic component of complex phenotypes compared to random genetic variation. Based on 500,000 individuals from the UK Biobank, we found the estimated contribution of Neanderthal alleles (NIMs) to phenotypic variation (NIM heritability) is significantly depleted in the great majority of the phenotypes. This is consistent with the observation that in general, natural selection has acted to remove Neanderthal alleles since introgression. On the other hand, we have found that Neanderthal alleles were significantly over-represented in their contribution to a handful of traits.
To understand the evolutionary models that could explain these observations, we performed forward-in-time population genetic simulations to model the evolution of Neanderthal and non-Neanderthal alleles according to a demographic model relating modern humans and Neanderthals. We chose parameters used in a previous study (Petr PNAS 2019) analyzing the fitness cost of Neanderthal introgression. Specifically, an ancestral population of size 10,000 diploid individuals splits into a human population and a Neanderthal population, each one evolves separately before a single pulse of Neanderthal admixture followed by subsequent random mating. Under this demography, we modeled evolution of phenotypes subject to different forces including directional, stabilizing, and disruptive selection. We estimated a NIM heritability Z-score, a measure of whether NIM heritability deviates significantly from the background alleles. We found under most models of selection, the NIM heritability Z-score is near zero or negative, indicating NIM heritability is neutral or depleted. Interestingly, we were able to recreate a positive NIM heritability Z-score, indicating an elevated Neanderthal contribution to heritability in two separate models of stabilizing and directional selection. In the stabilizing selection model, the optimal value of the trait is decreased in the human branch during the split between humans and Neanderthals leading to a positive NIM heritability Z-score. We also observe a positive NIM heritability Z-score in a directional selection model in which the parameter that couples SNP effect size and fitness is reduced after introgression. This observation highlights possible mechanisms for how complex traits evolved in human history by examining the genetic contribution of Neanderthal ancestry.
