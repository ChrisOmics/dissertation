\chapter{Impact of Neanderthal DNA on Depression}

\section{Introduction}
Genetic Relationship with Archaic Hominin Individuals
Past studies of Neandertal genomes have shown that the East Asians have inherited ∼20\% more Neandertal ancestry than Europeans and that this excess ancestry may reflect a second pulse of admixture in East Asians or a dilution of Neandertal admixture in Europeans (Prufer et al. 2014; Sankararaman et al. 2014, 2016; Vernot and Akey 2014, 2015; Kim and Lohmueller 2015). We largely recapitulated the relationship of a number of Neandertal samples and Denisovan to the Han Chinese as previously reported (supplementary fig. S9, Supplementary Material online). We observed subtle differences in allele-sharing pattern and estimated Neandertal ancestry (∼1.8–2\%) across China, though the difference is not significant after correcting for multiple testing (supplementary table S8, Supplementary Material online).
\section{Methods}
Previous analyses of the locations of Neandertal segments within the genomes of non-African individuals indicated that some of the Neandertal variants were adaptively beneficial while the bulk of Neanderthal variants were deleterious in the modern human genetic background (Harris and Nielsen 2016; Juric et al. 2016). Specifically, a recent examination of Neandertal-informative markers (NIMs) among large cohort of Europeans showed that these markers explained some proportions of the phenotypic risk of a number of diseases in the electronic health record (Simonti et al. 2016), including MDD. We sought to replicate this finding in East Asians as our data set was originally ascertained as a case–control study of MDD in Han Chinese women (Cai et al. 2015).

We extracted 75,539 SNPs that were previously identified to tag Neandertal haplotypes in East Asian individuals in the 1KG project (Sankararaman et al. 2014), and assessed the contribution of these NIMs to depression in our cohort consisting 5,224 cases of MDD and 5,218 controls. The allele frequencies of these NIMs are highly correlated $(r=0.951)$ between our cohort and 1KG, suggesting that the NIMs are not overt outliers from the rest of the variants in our data set in terms of data quality. We tested the association between the NIMs and depression by performing a logistic regression of depression, controlling for age and the first ten PCs, for MDD and Melancholia. Using the current sample size and sequence data, we found no association surviving the Bonferroni correction (supplementary fig. S10, Supplementary Material online) and the QQ plots did not reveal any systematic inflation nor significant enrichment among top associated SNPs (data were not shown).

We also calculated the proportion of phenotypic variance explained by these NIMs using GCTA (Yang et al. 2011) for MDD. We used a prevalence of 7.5\% to transform the heritability to the liability scale. We found that the variance explained by the NIMs is ∼1\%, which is different from that reported in Simonti et al. (∼2\%) and is not significantly different from 0 $(P=0.12)$. Repeating the analysis with NIMs with MAF >0.01 as well as with no covariates did not qualitatively alter the results (supplementary table S9, Supplementary Material online). Finally, we found that the heritability explained by NIMs is not significantly different from that of a background set of SNPs chosen at random to match the NIMs by derived allele frequency decile and by Linkage Disequilibrium (LD) scores $(P>0.4)$. Our analysis may be underpowered given the smaller sample size and low coverage, but the results could suggest that the impact of Neandertal ancestry on MDD differs between European and Han Chinese. Future investigation in larger cohorts will be informative.
%***********PBC***************
\section{PBC}
